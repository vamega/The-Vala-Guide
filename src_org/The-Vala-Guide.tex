\documentclass[openany]{book}

\usepackage{fontspec}
\usepackage{color}
\usepackage{fancyvrb}
\usepackage{minted}
\usepackage{url}

\setromanfont{Adobe Garamond Pro}
\setsansfont{Optima LT Std}
\setmonofont{Nimbus Mono L}

%%------------------------------------------------------------
% formatting commands

%\sloppy
\setlength{\topmargin}{0.125in}
\setlength{\oddsidemargin}{0.25in}
\setlength{\evensidemargin}{0.25in}
\setlength{\textwidth}{430pt}

\setlength{\headsep}{3ex}
\setlength{\textheight}{8in}

\setlength{\parindent}{0.0in}
\setlength{\parskip}{1.7ex plus 0.5ex minus 0.5ex}
\renewcommand{\baselinestretch}{1.02}

% see LaTeX Companion page 62
\setlength{\topsep}{-0.0\parskip}
\setlength{\partopsep}{-0.5\parskip}
\setlength{\itemindent}{0.0in}
\setlength{\listparindent}{0.0in}

% see LaTeX Companion page 26
% these are copied from /usr/local/teTeX/share/texmf/tex/latex/base/book.cls
% all I changed is afterskip

\makeatletter
\renewcommand{\section}{\@startsection 
    {section} {1} {0mm}%
    {-3.5ex \@plus -1ex \@minus -.2ex}%
    {0.7ex \@plus.2ex}%
    {\normalfont\Large\bfseries}}
\renewcommand\subsection{\@startsection {subsection}{2}{0mm}%
    {-3.25ex\@plus -1ex \@minus -.2ex}%
    {0.3ex \@plus .2ex}%
    {\normalfont\large\bfseries}}
\renewcommand\subsubsection{\@startsection {subsubsection}{3}{0mm}%
    {-3.25ex\@plus -1ex \@minus -.2ex}%
    {0.3ex \@plus .2ex}%
    {\normalfont\normalsize\bfseries}}

\makeatother

\begin{document}

\renewcommand{\theFancyVerbLine}{
  \bfseries\ttfamily\textcolor[RGB]{32,32,32}{\scriptsize\arabic{FancyVerbLine}}}

\definecolor{bg}{RGB}{32,32,32}
\usemintedstyle{native}

%--title page--------------------------------------------------
\pagebreak
\thispagestyle{empty}

\begin{flushright}
\vspace*{2.5in}

{\huge The Vala Guide}

\vspace{1in}

{\Large
Varun Madiath
}


\vspace{1in}

{\Large Version 0.1}

{\small \today}

\vfill

\end{flushright}

%--copyright--------------------------------------------------
\pagebreak
\thispagestyle{empty}

Copyright \copyright ~2003, 2008 Allen Downey.

\vspace{0.25in}

Permission is granted to copy, distribute, and/or modify this document
under the terms of the GNU Free Documentation License, Version 1.1 or
any later version published by the Free Software Foundation; with
Invariant Sections being ``Preface'', with no Front-Cover Texts, and
with no Back-Cover Texts.  A copy of the license is included in the
appendix entitled ``GNU Free Documentation License.''

The GNU Free Documentation License is available from \url{www.gnu.org}
or by writing to the Free Software Foundation, Inc., 59 Temple Place,
Suite 330, Boston, MA 02111-1307, USA.

The original form of this book is \LaTeX\ source code.  Compiling this
\LaTeX\ source has the effect of generating a device-independent
representation of the book, which can be converted to other formats
and printed.

The \LaTeX\ source for this book is available from

\begin{verbatim}
      http://github.com/vamega/The-Vala-Guide
\end{verbatim}

\vspace{0.25in}

%-- Chapters Begin-------------------------
\chapter{Some Background}
Vala is a new programming language that aims to bring modern programming language features to GNOME developers without imposing any additional runtime requirements and without using a different ABI compared to applications and libraries written in C.

valac, the Vala compiler, is a self-hosting compiler that translates Vala source code into C source and header files. It uses the GObject type system to create classes and interfaces declared in the Vala source code.

The syntax of Vala is similar to C\#, modified to better fit the GObject type system. Vala supports modern language features as the following:

\begin{itemize}
	\item Interfaces
	\item Properties
	\item Signals
	\item Foreach
	\item Lambda expressions
	\item Type inference for local variables
	\item Generics
	\item Non-null types
	\item Assisted memory management
	\item Exception handling
\end{itemize}

Vala is designed to allow access to existing C libraries, especially GObject-based libraries, without the need for runtime bindings. All that is needed to use a library with Vala is an API file, containing the class and method declarations in Vala syntax. Vala currently comes with bindings for GLib and GTK+. It's planned to provide generated bindings for the full GNOME Platform at a later stage.

Using classes and methods written in Vala from an application written in C is not difficult. The Vala library only has to install the generated header files and C applications may then access the GObject-based API of the Vala library as usual. It should also be easily possible to write a bindings generator for access to Vala libraries from applications written in e.g. C\# as the Vala parser is written as a library, so that all compile-time information is available when generating a binding.

More information about Vala is available at

	http://live.gnome.org/Vala/

\inputminted[linenos=true,bgcolor=bg]{vala}{src/main.vala}


\end{document}

